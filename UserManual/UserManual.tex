\documentclass{article}

\usepackage{graphicx}
\usepackage[spanish]{babel}
\usepackage[margin=2cm]{geometry}

\setlength{\parindent}{2em}
\setlength{\parskip}{1em}

\begin{document}
    
\begin{titlepage}
    \begin{center}
        \includegraphics{img/Logo UAM.png}\par\vspace{1cm}\par\vspace{1cm}
        {\scshape\LARGE Universidad Autónoma de Madrid \par}
        \vspace{1cm}
        {\scshape\Large Prácticas PSI 2020\par}
        \vspace{1.5cm}
        {\huge\bfseries Manual de Usuario\par}
        \vspace{4cm}
        {\Large\itshape Adrián Sebastián Gil\par}
        {\Large\itshape Javier Mateos Najarí\par}
        \vspace{2cm}
        {\large Grupo 1321\par}
        \large\today
    \end{center}
\end{titlepage} 

\section*{Usuario sin Iniciar sesión}

Al abrir la página sin inciar sesión aparece en la parte superior de la ventana un
\textit{header} con los enlaces \textit{Home} y \textit{Log In} y un 
\textit{footer} con el nombre de los autores de la página. Esto será común para todas
las páginas mientras el usuario siga sin inciar sesión.

\subsection*{Home Page}

En la página home hay una lista de enlaces dónde se describe la funcionalidad
de la página siguiendo el orden del flujo lógico de uso que se debería hacer en la
aplicación. Cada enlace lleva a diferentes páginas dónde se describe como realizar 
cualquier acción dentro de la página.

El primer enlace lleva a la descripción del login dónde se explica los datos necesarios
para inciar la sesión.

El segundo lleva a la página dónde se describen los requisitos para optar a convalidar las
prácticas de la asignatura.

El tercero lleva a un enlace donde se explica desde dónde y cómo se crean las parejas y las
implicaciones que esto tiene.

Y por útimo, está la ventana dónde se describe cuando, cómo y que implicaciones tiene
tener pareja a la hora de elegir grupo.


\subsection*{Log In Page}

En la página de \textit{login} debes introducir tus credenciales y darle a \textit{Submit}. Las credenciales
son el NIE como \textit{Username}  y el DNI como \textit{Password}.


\section*{Usuario Autenticado}

Una vez que el usuario ha iniciado sesión, en el \textit{header} tiene enlaces para realizar
las diferentes acciones posibles en la página. Estos enlaces son:


\subsection*{Home page}

En la página inicial se da la información del alumno con respecto a la asignatura:
\begin{itemize}
    \item Nombre y grupo de teoría.
    \item Estado de convalidación de las prácticas.
    \item Estado de la pareja.
    \item Estado del grupo de prácticas 
\end{itemize}


\subsection*{Logout Page}

Cierra automáticamente la sesión y vuelves al estado de usuario sin iniciar sesión explicado previamente.


\subsection*{Convalidation Page}

Lleva a la página de convalidación dónde te dice si eres apto o no para convalidar y se lleva a cabo en caso
que sea posible.


\subsection*{Apply Group Page}

Muestra un desplegable con las posibilidades de elección para el grupo de prácticas, en caso de que ya hayas
elegido o se te haya asignado un grupo te muestra el grupo al que perteneces.

Si tienes una pareja, el grupo se asigna a ambos miembros de esta.


\subsection*{Apply Pair Page}

Dentro de esta página hay una breve descripción sobre como validar una pareja, esto se hace eligiendo cada miembro
de la pareja a su compañero desde esta ventana. Una vez ambos alumnos han seleccionado a su compañero la pareja pasa 
a estar validada.

Para seleccionar compañero hay un menu desplegable donde se muestran todos los compañeros disponibles.


\subsection*{Break Pair Page}

En esta sección hay una breve descripción de como romper una pareja previamente creada, esto soló se puede
hacer cuando todavía no se tiene un grupo de laboratorio asignado.

En caso de no tenerlo hay dos posibilidades, una cuando la pareja esta validada y otra cuando no. Para las parejas
no validadas basta con que uno de los dos miembros de la pareja soliciten la ruptura para que esta se
lleve a cabo. En el caso de que la pareja este validada ambos miembros deberán solicitar la ruptura de la pareja para
que esta se realice.


\subsection*{Help Page}

Esta página contiene la misma guía de uso que la página \textit{home} para usuarios sin la sesión iniciada exceptuando 
la ayuda para realizar el inicio de sesión.



\end{document}